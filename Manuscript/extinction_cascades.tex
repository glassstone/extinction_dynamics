%--------------------------------------||--------------------------------------%

\documentclass[11pt,letterpaper]{article} %  font size


%-----------------------------------PACKAGES-----------------------------------%

\usepackage[T1]{fontenc} % Choose an output font encoding (T1) that has support for the accented characters used by the most widespread European languages
\usepackage[utf8]{inputenc} % Allow input of accented characters (and more...)
\usepackage{graphicx} %  figures
\usepackage[round]{natbib} % names in citations
\usepackage{lineno} % line numbers
\usepackage{authblk} % allows more intuitive formatting for multiple authors/affiliations
\usepackage[margin=1in]{geometry} % make the margins 1 inch on all sides of the document
%\usepackage{amsmath} % useful for formatting math stuff, especially complex equations
%\usepackage{pdflscape} % rotate table into landscape mode
%\usepackage{subfigure} % side by side figures
%\usepackage{longtable} % for tables that span multiple pages.
\usepackage{setspace} % double spacing



%----------------------------------FORMATTING-----------------------------------

%\topmargin -1cm %0.0cm
%\textwidth 16cm % what does this do?
%\textheight 21cm % what does this do?
%\footskip 1.0cm % what does this do?
%\oddsidemargin 0.0cm

\date{\today}
\doublespacing % initiate double spacing (package setspace)
%\linespread{2} % alternate method of double spacing


%------------------------------------TITLE--------------------------------------

\title{This is the new and improved title}

%-----------------------------------AUTHORS-------------------------------------

% using package 'authblk':
\author[1]{First Author\thanks{first.author@funstuff.com}}

\author[1,2]{Second Author}

\author[2]{Third Author}

\affil[1]{Department of Computer Science, \LaTeX\ University}
\affil[2]{Department of Mechanical Engineering, Superfabulous University}

% This is a janky way of doing authors, but avoids extra packages if you're into that.
%\author{
%	First Author\\
%		\small{\textit{First Author Affiliation}}\\
%		\small{\textit{firstauthor@affiliation.com}} \and
%	Second Author\\
%		\small{\textit{Second Author Affiliation}}\\
%		\small{\textit{secondauthor@affiliation.com}}
%	}


%----------------------------------FORMATTING-----------------------------------
\begin{document}
\maketitle
\linenumbers % start line numbers
\def\linenumberfont{\normalfont\small\rmfamily} % change line number font


%----------------------------------KEYWORDS-------------------------------------
\section*{Keywords}
Stuff, things, neat, cool, wow, instafun, tags4likes, etc

%----------------------------------ABSTRACT-------------------------------------
\section*{Abstract}
This is the text of the abstract.

%---------------------------------INTRODUCTION----------------------------------
\section*{Introduction}
For centuries, humankind has wondered: If I have two apples, and someone gives me another two apples, how many apples do I have? Some people did this \citep{Darwin1859}

One example of a species with a low dependence on a mutualistic network (R) is plants that have the ability to self-pollinate. 


%-----------------------------------METHODS-------------------------------------
\section*{Methods}
We modeled the distribution of plants' reliance on pollination after empirical assessment of the rate of selfing and outcrossing for 169 animal-pollinated species (CITE Vogler and Kalisz 2001 EVO). This distribution is multimodal; thus we drew from a binomial distribution to subsequently drew from either of two beta distributions with shape parameters 0.8 and 2 or 13 and 4.

A species was removed at random (unless we think it's important to choose based on either a species' R (dependence on network) or it's degree (number of interactions) or H' (distribution of interaction strengths).

We conducted simulations on XXX networks collected primarily from the Interaction Web Database ?(NCEAS? CITATION?).



We constructed the following mathematical model (\ref{my_equation_label}) to better understand the concept:
% using '\ref{something}' in the text will refer to any object (e.g. figure, equation, table) which contains the corresponding '\label{something}'
% Note that you need to run latex a few times to get it to register numbers correctly

\begin{equation}\label{my_equation_label}
	Y = 2a + 2a
\end{equation}

Where $a$ represents an apple.

%-----------------------------------RESULTS-------------------------------------
\section*{Results}
We found that if you have two apples, and someone gives you another two apples, you have four apples.

%----------------------------------DISCUSSION-----------------------------------
\section*{Discussion}
Boy those results sure are neat. Now, the pressing question becomes: How do you like them apples?

%-------------------------------ACKNOWLEDGEMENTS--------------------------------
\section*{Acknowledgements}
We wish to thank all of the little people.

%-----------------------------------FUNDING-------------------------------------
\section*{Funding}
This study was funded by our super-rich uncle.

%--------------------------------CONTRIBUTIONS----------------------------------
\section*{Author Contributions}
Conceived and designed the experiments: .
Collected the data: .
Conducted the analyses: .
Wrote the first draft: .
Edited the manuscript: .


%-------------------------------------DATA-------------------------------------%
\section*{Data Availablity}
% Are the data and code available in a permanent, publicly accessible data archive or repository?
The data and code used to generate our results can be found at the following url: 


%----------------------------------REFERENCES-----------------------------------
%\section*{References} % commented out because the section title is automatically inserted if using an automatically-generated bibliography

\bibliographystyle{apalike} % or: plain,unsrt,alpha,abbrv,acm,apalike,ieeetr
\bibliography{/Users/threeprime/Compute/Tex/Templates/article_default_coolrefs} % path to your .bib file excluding .bib extension (e.g. /Users/threeprime/Documents/Publications/bibtex/library)


%-----------------------------------FIGURES-------------------------------------
\section*{Figures}

%\begin{figure}[h!] % [h!] forces the figure to be placed roughly here
%  \centering
%    \includegraphics[width=1\textwidth]{figure_filename.pdf}
%    \caption{This is the figure caption.}
%  \label{myfigure} % use this to refer to your figure in the text, so that numbering updates automatically
%\end{figure}


%----------------------------------SUPPLEMENT-----------------------------------
%\section*{Supplemental Material}





\end{document}