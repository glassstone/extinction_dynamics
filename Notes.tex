\documentclass[12pt]{article}

\begin{document}


\subsubsection*{Background}
Vieira and Almeida-Neto (2014) proposed a stochastic coextinction model to simulate extinction cascades with more complexity than previous models. One major finding contrary to existing (topological coextinction) models (TCM) was that the likelihood and severity (degree; ``the number of extinction episodes, with each episode involving one or more species, summed across both tropic levels'') of a cascade increases with network connectance in mutualistic networks. Their model (1) picked species at random for the primary extinction event, and (2) incorporated information about the species' dependence on the interaction, $R$, but drew this value randomly from a uniform distribution, independent of other properties of the species, and applied the same value to all species. Importantly, they showed that $R$ chosen from either low, medium, or high values has substantial impact on the outcome of the extinction cascade (probability of a cascade, mean number of extinctions, proportion spp lost). However, we know that not all species are equally prone to extinction, and the dependence of a species on an interaction network can covary with other properties of the species. We build upon their model by modifying the parameter $R$, with the goal of making coextinction simulations more realistic by incorporating information from ecological networks and conducting simulations on both empirical and simulated datasets. We report how the covariation between species' network traits (e.g. degree, strength, specialization) and their dependence on a network $R$ impacts coextinction cascades. Further, we assess differences in network structure and species' traits before and after extinction cascades.
Further, we test our model on simulated networks generated according to parameters that people seem to think are important properties of mutualistic networks. 

What are the properties of all of the species; esp - are there differences between the primary extinction and secondary extinction. Is there similarity in network traits of species that never go extinct? 

Previous studies made the following assumptions:
\begin{enumerate}
	\item Coextinction requires the loss of all partners
	\item Interaction strengths do not matter (i.e. a species survives as long as at least one partner survives, even if if only weakly interacts with that partner)
	\item Variation in how much a species depend on their partners was ignored. For example, plants' ability to self-pollinate is ignored.
\end{enumerate}

Thus, Vieira and Almeida-Neto proposed a model that
\begin{enumerate}
	\item incorporates variation in dependences
	\item relaxes the assumption that coextinction requires the loss of all interaction partners
\end{enumerate}

Under the proposed model the dependence of species on the interactions controls the likelihood of extinction cascades and thus highly connected networks are more susceptible to cascades. The model is more realistic since it considers the possibility of compensatory effects.\\

Their model:

Let:\\
$d_{ij}$ is the population-level strength of the interaction.\\
$R_i$ is assumed constant for each species and reflects the intrinsic demographic dependence of species $i$ on the mutualism in question\\
$P_{ij} = R_i d_{ij}$ is the probability of species $i$ going extinct following the extinction of a mutualistic partner species $j$.\\

Simulations proceed by removing one species at a given trophic level, and removing species from the other set according to $P$. $d_{ij}$ are recalculated after each extinction


\subsubsection*{Possible Directions}

\begin{enumerate}

	\item MP originally stated that Vieira and Almeida-Neto consider only a single value of $R$ for the entire network; however, JO believes each species has a different $R$. Note that each species has ONE $R$, not a different $R$ for each of the possible partners. MP: This is unrealistic. Different species depend on the interaction with different degrees. This R could also be seen as the participation of the species in a particular network. For instance, if we are studying a frugivory network species that eat insects, but are also included in the network will have smaller $R$ than strict frugivores. Similarly, a species that feed on fruits that we are considering but also on other fruits that, for some reason, are not included in this particular network, has smaller $R$.  Thus, the $R$ parameter is what defines the boundaries of the network. It would be interesting to perform simulation where $R$ obeys a probability distributions. How different $R$ distributions changes the size and shape of extinction cascades? Are there critical points given by the parameters of these distributions?

	\item The authors only explored mutualisms. It would be interesting to test whether their conclusions hold for other interaction types.

	\item One of the possibilities we discussed was testing the effects of extinction on multilayered networks. This would be the obvious step after step 1. We could start from the simple scenario with 2 interaction types (e.g. mutualism, herbivory) and then explore other combinations. An empirical system that we could play with are defensive mutualisms for which MP believes Cecilia might have suitable data.

	\item Since the model was designed with mutualisms in mind, one extinction can only have negative effects over the partners. However when we think on antagonisms (competition, predation, parasitism), extinctions may favor other species. Since the model is probabilistic, we could also have, Q, which is the probability of density increase after an extinction. P could also change if antagonists increased enough in abundance. we should discuss the feasibility of adding this to the model while keeping it simple.

\end{enumerate}

\subsubsection*{What we've done}

\begin{enumerate}
	\item{Generalized the model so each species has an R value}
	\item{R values are sampled from a probability distribution}
	\item{Exploring how extinctions cascades respond to R distributions with different shapes}	
\end{enumerate}







\subsubsection*{Next steps/TODO}

\begin{itemize}

	\item play with the R codes in the Supporting information and try to devise an alternate model encompassing option 4 above (extension to antagonisms)
	
	\item think how we can generalize this to antagonist interactions. In antagonistic interactions the loss of a prey species may have a demographic impact on the predator, but the loss of the predator can have a positive impact on the prey - We need to discuss how to generalize the same model to include these possibilities


	\item First test - How including variable Rvalues (species dependence on interactions) across species change the original results? Use the same network, same rlow, same rup, one set of simulations with one Rvalue (as in Vieira and Almeida-Neto 2015), second set of simulation with several Rvalues

	\item Second test - How different distributions of Rvalues (different ways in which species depend on the mutualism) change the probability of cascades and the number of coextinctions? We could choose uniform, normal and exponential, and few parameter combinations for each (3?)

	\item Third test - Does network topology affect the consequence of multiple Rvalues? Perform a series of tests with networks with different topologies

	\item for all tests besides looking at the number of coextinctions and coextinctions degree we could examine how topology changed

\end{itemize}

\subsubsection*{More thoughts~\ldots}

From meeting 20150818 (MP+JO):

$P_{ij} = R_{ij} d_{ij}$ , if $i$ is a predator\\

Probability that $i$ increases in density is given by:
$Q_{ij} = R_{ij} d_{ji}$, if $i$ is a prey species 

if the density increases we could have one iteration step where interaction frequencies are changed by some factor - assuming that population density affects interaction frequencies



A. response of model output to probabilistic values of R\\
B. behavior of model on non-mutualistic interactions\\
C. merge multiple networks using multi-layer network approach (perhaps very difficult)




% Important notes from Vieira:
%Ri was assumed to be equal for all species and was uniformly sampled in each simulation from three intervals representing low (0 < Ri ? 0.3), intermediate (0.3 < Ri ? 0.6) and high (0.6 < Ri ? 1) intrinsic demographic dependence on the mutualistic interaction for persistence. For each network, we performed 104 simulations for each interval of Ri and constructed empirical frequency distributions for the total number of extinctions in an extinction cascade. We also quantified the degree of each extinction cascade and constructed its corresponding frequency distribution. From this frequency distribution, we calculated, for each network and Ri level, the probability that a primary extinction would result in second-, third- and fourth-degree-or-higher extinction cascades.








\end{document}